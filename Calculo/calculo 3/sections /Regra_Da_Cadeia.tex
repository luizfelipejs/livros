\subsection{Regra Da Cadeia e Derivação implícita}

\begin{theorem}
	Supondo que $z = f(x, y)$ seja uma função diferenciável, onde $x = g(t)$ e $y = h(t)$ são funções diferenciáveis de $t$, então $z$ é uma função diferenciável de $t$, e segue que
	
	\[
	\dfrac{dz}{dt} = \dfrac{\partial f}{\partial x} \dfrac{dx}{dt} + \dfrac{\partial f}{\partial y} \dfrac{dy}{dt}
	\]
\end{theorem}

\begin{proof}
	Seja $z = f(x, y)$ uma função diferenciável, com $x = g(t)$ e $y = h(t)$ funções diferenciáveis de $t$. Então, ao variar $t$ em uma pequena quantidade $\Delta t$, temos variações $\Delta x$ em $x$ e $\Delta y$ em $y$, que por sua vez causam uma variação $\Delta z$ em $z$.
	
	Como $f$ é diferenciável, temos que a variação $\Delta z$ pode ser aproximada por:
	
	\[
	\Delta z \approx \frac{\partial f}{\partial x} \Delta x + \frac{\partial f}{\partial y} \Delta y
	\]
	
	Dividindo ambos os lados por $\Delta t$, obtemos:
	
	\[
	\frac{\Delta z}{\Delta t} \approx \frac{\partial f}{\partial x} \frac{\Delta x}{\Delta t} + \frac{\partial f}{\partial y} \frac{\Delta y}{\Delta t}
	\]
	
	Tomando o limite quando $\Delta t \to 0$, e usando a continuidade e a diferenciabilidade de $x(t)$ e $y(t)$, obtemos:
	
	\[
	\frac{dz}{dt} = \frac{\partial f}{\partial x} \frac{dx}{dt} + \frac{\partial f}{\partial y} \frac{dy}{dt}
	\]
	
	Portanto, a derivada de $z$ em relação a $t$ é dada pela fórmula da Regra da Cadeia.
\end{proof}

\begin{definition}[Regra da Cadeia — Caso 2]
	Suponha que $z = f(x, y)$ seja uma função diferenciável de $x$ e $y$, onde $x = g(s, t)$ e $y = h(s, t)$ são funções diferenciáveis de $s$ e $t$.
	
	Então, $z$ é uma função diferenciável de $s$ e $t$, e temos:
	
	\[
	\frac{\partial z}{\partial s} = \frac{\partial f}{\partial x} \frac{\partial x}{\partial s} + \frac{\partial f}{\partial y} \frac{\partial y}{\partial s}
	\quad \text{e} \quad
	\frac{\partial z}{\partial t} = \frac{\partial f}{\partial x} \frac{\partial x}{\partial t} + \frac{\partial f}{\partial y} \frac{\partial y}{\partial t}
	\]
\end{definition}

\begin{theorem}
	Suponha que $u$ seja uma função diferenciável de $n$ variáveis $x_1, x_2, \dots, x_n$, onde cada $x_j$ é uma função diferenciável de $m$ variáveis $t_1, t_2, \dots, t_m$. Então $u$ é uma função diferenciável de $t_1, t_2, \dots, t_m$ e, para cada $i = 1, 2, \dots, m$, temos:
	
	\[
	\frac{\partial u}{\partial t_i} 
	= 
	\frac{\partial u}{\partial x_1} \frac{\partial x_1}{\partial t_i}
	+ \frac{\partial u}{\partial x_2} \frac{\partial x_2}{\partial t_i}
	+ \dots 
	+ \frac{\partial u}{\partial x_n} \frac{\partial x_n}{\partial t_i}
	\]
	
	ou, de forma compacta:
	
	\[
	\frac{\partial u}{\partial t_i} 
	= 
	\sum_{j=1}^{n} \frac{\partial u}{\partial x_j} \frac{\partial x_j}{\partial t_i}
	\]
\end{theorem}
